\documentclass[a4paper, 12pt, twoside=false]{scrbook}

\usepackage[utf8]{inputenc}
\usepackage[english]{babel}
\usepackage{csquotes}
\usepackage{amsthm}
\usepackage{amsmath}
\usepackage{amssymb}
\usepackage{enumerate}
\usepackage[colorlinks=true]{hyperref}
\usepackage{makeidx}

\makeindex

\usepackage[backend=biber]{biblatex}
\addbibresource{sample.bib}

\theoremstyle{definition}
\newtheorem*{definition}{Definition}
\newtheorem*{remark}{Remark}
\newtheorem{theorem}{Theorem}[chapter]
\newtheorem{corollary}[theorem]{Corollary}
\newtheorem{lemma}[theorem]{Lemma}
\newtheorem{proposition}[theorem]{Proposition}

\newcommand*{\id}{\ensuremath{\mathrm{id}}}
\newcommand*{\IC}{\ensuremath{\mathbb{C}}}
\newcommand*{\IR}{\ensuremath{\mathbb{R}}}
\newcommand*{\IQ}{\ensuremath{\mathbb{Q}}}
\newcommand*{\IZ}{\ensuremath{\mathbb{Z}}}
\newcommand*{\IN}{\ensuremath{\mathbb{N}}}
\let\eps\varepsilon

\title{Intersecting clutters}
\author{Martin Drees}
\begin{document}
   \maketitle
   \tableofcontents
   \chapter{Introduction}
   %TODO Motivation
   This thesis is heavily based on \textquote{Intersecting restrictions in clutters} by Abdi, Cornuéjols and Lee \cite{restrictions}.
   In this first chapter some basic concepts and tools around clutters are introduced mainly following the introduction of \cite{restrictions}.
   \section{Clutters}
   \begin{definition}
       A \emph{clutter}\index{clutter} over a finite ground set $V$ is a family of subsets of $V$ called members such that no member contains another.
       The clutters $\{\}$ and $\{\emptyset\}$ are called \emph{trivial} clutters.
       Two clutters are \emph{isomorphic}\index{isomorphic} if one can be obtained from the other by relabeling the ground set.
   \end{definition}
   The following lemma is a useful tool to check whether to clutters are identical.
   \begin{lemma}\label{equalclutters}
       Let $\mathcal{C}_1$ and $\mathcal{C}_2$ be two clutters over the same ground set $V$ such that every member of $\mathcal{C}_1$ contains a member of $\mathcal{C}_2$ and vice versa.
       Then $\mathcal{C}_1=\mathcal{C}_2$.
   \end{lemma}

   \begin{proof}
       Let $C_1 \in \mathcal{C}_1$ be an arbitrary member.
       We find a member $C_2 \in \mathcal{C}_2$ with $C_1 \supseteq C_2$.
       Using $C_2$ we find a member $C_1' \in \mathcal{C}_1'$ such that $C_1 \supseteq C_2 \supseteq C_1'$.
       Since $\mathcal{C}_1$ is a clutter we get $C_1=C_1'$ implying $C_1=C_2$.
       We get $\mathcal{C}_1 \subseteq \mathcal{C}_2$.
       Similarly, $\mathcal{C}_2 \subseteq \mathcal{C}_1$, implying $\mathcal{C}_1=\mathcal{C}_2$.
   \end{proof}

\section{Covers and packings}
\begin{definition}
    Let $\mathcal{C}$ be a clutter over ground set $V$.
    A \emph{cover}\index{cover} of $\mathcal{C}$ is a set $B \subseteq V$ such that $B \cap C \neq \emptyset$ for all members $C \in \mathcal{C}$.
    A cover is called \emph{minimal} if it does not contain another cover.
    The \emph{covering number}\index{covering number} $\tau(\mathcal{C})$ is the minimum cardinality of a cover.
\end{definition}

\begin{definition}
    A \emph{packing}\index{packing} of a clutter is a set of pairwise disjoint members.
    The \emph{packing number}\index{packing number} $\nu(\mathcal{C})$ is the maximum cardinality of a packing.
\end{definition}

For a nontrivial clutter $\mathcal{C}$ we have $\tau(\mathcal{C}) \geq \nu(\mathcal{C})$ since each of the pairwise disjoint members in a packing contains one element of a cover.

\section{Blockers}
\begin{definition}
    Let $\mathcal{C}$ be a clutter over ground set $V$.
    The \emph{blocker}\index{blocker} $b(\mathcal{C})$ is the clutter over ground set $V$ with the minimal covers of $\mathcal{C}$ as members.
\end{definition}

%TODO Reference
\begin{lemma}[Isbell 1958]
    A clutter $\mathcal{C}$ satisfies $b(b(\mathcal{C}))=\mathcal{C}$.
\end{lemma}

\begin{proof}
    Let $V$ be the ground set of the clutter.
    Let $C \in \mathcal{C}$.
    By the definition of the blocker, for each $B \in b(\mathcal{C})$ we have $B\cap C \neq \emptyset$.
    Therefore $C$ is a cover of $b(\mathcal{C})$ and thus contains a minimal cover.
    So every member of $\mathcal{C}$ contains a member of $b(b(\mathcal{C}))$.

    Now let $C$ be a minimal cover of $b(\mathcal{C})$.
    So $V-C$ does not contain a member of $b(\mathcal{C})$, implying that $V-C$ is not a cover of $\mathcal{C}$.
    Therefore $C$ contains a member of $\mathcal{C}$.
    Hence every member of $b(b(\mathcal{C}))$ contains a member of $\mathcal{C}$.

    \hyperref[equalclutters]{Lemma \ref*{equalclutters}} implies $\mathcal{C}=b(b(\mathcal{C}))$.
\end{proof}


\section{Minors and restrictions}
\begin{definition}
    Let $\mathcal{C}$ be a clutter over ground set $V$ and $I, J \subseteq V$ disjoint subsets.
    The \emph{minor}\index{minor} of $\mathcal{C}$ obtained after \emph{deleting} $I$ and \emph{contracting} $J$ is the clutter $\mathcal{C} \backslash I / J$ over ground set $V - (I \cup J)$ whose members are the inclusion-wise minimal sets of $\{C-J : C~\in~\mathcal{C},\, C~\cap~I = \emptyset\}$.
    If $I \cup J \neq \emptyset$ the minor is called \emph{proper}.
\end{definition}

\begin{definition}
    Let $\mathcal{C}$ be a clutter over ground set $V$ and $I \subseteq V$ such that $I$ is not a cover of $\mathcal{C}$.
    Let
    \begin{align*}
        J := \{u \in V-I : \{u\} \text{ is a cover of } \mathcal{C} \backslash I\} \;.
    \end{align*}
    The minor $\mathcal{C} \backslash I / J$ is the \emph{restriction}\index{restriction} of $\mathcal{C}$ after restricting $I$.
\end{definition}

By the definition of the set $J$, a restriction is either trivial or $\tau(\mathcal{C}) \geq 2$ since a cover of size 1 would have been contracted.

Since $I$ is not a cover, for every member $C \in \mathcal{C} \backslash I /J$ the set $C \cup J$ is a member of $\mathcal{C}$.

\begin{remark}
    In the definition of a restriction in \cite{restrictions} by Abdi, Cornuéjols and Lee, it is allowed to restrict a cover.
    The reason for the additional constraint is that we are often interested in the existence of members with certain properties.
    If all members of the clutter fulfill that property we would need to prove that there is at least a member and consider the case anyway.
\end{remark}


\section{Intersecting clutters}
\begin{definition}
    A clutter is \emph{intersecting} if $\tau(\mathcal{C}) > \nu(\mathcal{C}) = 1$, i.e. every two members intersect, but not all members have a common element.
    The trivial clutters are not intersecting.
\end{definition}
We will now see a few examples of intersecting clutters that will also play a major role throughout this entire work.

%TODO Pictures

Take the clutter over ground set $\{1,2,\ldots,6\}$ with members $\{1,3,5\}, \{1,4,6\}, \{2,3,6\}$ and $\{2,4,5\}$.
Note that this clutter can be represented as the triangles of $K_4$, the complete graph on four vertices whose edges are the ground set of this clutter.
This clutter is denoted by $Q_6$\index{$Q_6$} \cite{q6}.
There is no cover of size one since after deleting an arbitrary edge of $K_4$ a triangle is still remaining.
Furthermore there are no disjoint members since two disjoint triangles would use up all the edges and the parity at the vertices must be even, which is not the case.
Thus $\tau(Q_6) > \nu(Q_6) = 1$, so $Q_6$ is an intersecting clutter.

Let $n \geq 3$ be an integer.
Take a clutter over ground set $\{1,2,\ldots,n\}$ with members $\{1,2\},\{1,3\},\ldots,\{1,n\}$ and $\{2,3,\ldots,n\}$.
This clutter is called $\Delta_n$ and a clutter isomorphic to this clutter is called a \emph{delta of dimension $n$}\index{delta}\cite{deltadefinition}.
Observe that $b(\Delta_n)=\Delta_n$.
A delta clearly does not have disjoint members and $n\geq 3$ implies, that there is no cover of size 1.
Hence, deltas are intersecting clutters.

Let $n\geq 5$ be an odd integer.
Consider the clutter $\mathcal{C}$ over ground set $\{1,2,\ldots,n\}$ such that the minimum cardinality members of $\mathcal{C}$ are $\{1,2\},\{2,3\},\ldots,\{n-1,n\}$ and $\{n,1\}$.
A clutter that is isomorphic to $\mathcal{C}$ is an \emph{extended odd hole of dimension $n$}\index{extended odd hole}\cite{deltas}.
Obviously, an extended odd hole is not intersecting since there are disjoint members.
But the blocker of an extended odd hole actually is intersecting.
Every minimal cover of $\mathcal{C}$ has at least $\frac{n+1}2$ elements since $n$ is odd.
Therefore two minimal covers cannot be disjoint, so the blocker of an extended odd hole does not have disjoint members.
Moreover, the blocker of the blocker of an extended odd hole in an extended odd hole and has no member of size 1.
Therefore the blocker of an extended odd hole does not have cover of size 1.
Hence, blockers of extended odd holes are intersecting clutters.

\section{Dense clutters}
\begin{definition}
    Let $\mathcal{C}$ be a clutter over ground set $V$.
    A \emph{fractional packing}\index{fractional packing} is a vector $y \in \IR_{\geq0}^{\mathcal{C}}$ such that
    \begin{align*}
        \sum_{v \in C \in \mathcal{C}} y_C \leq 1 \quad \forall v \in V \,.
    \end{align*}
    The \emph{value} of the fractional packing is $\textbf{1}^Ty$.
\end{definition}
\begin{definition}
    A clutter $\mathcal{C}$ is called \emph{dense}, if $\tau(\mathcal{C}) \geq 2$ and $\mathcal{C}$ has no fractional packing of value 2.
\end{definition}

Observe that a dense clutter is also intersecting since two disjoint members would yield a fractional packing of value 2.
The converse is not necessarily true.
The clutter $Q_6$ of triangles in the $K_4$ has a fractional packing of value 2 by assigning $\frac 12$ to each of the four triangles, but is intersecting.

The following result gives a certificate for a dense clutter.
\begin{lemma}
    Let $\mathcal{C}$ be a clutter with $\tau(\mathcal{C})\geq 2$ over ground set $V$.
    Then the following are equivalent:
    \leavevmode
    \begin{enumerate}[(i)]
        \item $\mathcal{C}$ is dense,
        \item there is a $w \in \IR_{\geq 0}^V$ with $\textbf{1}^Tw=1$ such that $\sum_{u \in C} w_u > \frac 12$ for all C $\in \mathcal{C}$.
    \end{enumerate}
\end{lemma}

\begin{proof}
    Consider the following dual pair of linear programs
    \newline
    \newline
    \begin{minipage}{.5\linewidth}
        \begin{equation*}
            \begin{array}{l@{\quad} r l l}
                (P)\\
                \max          &z   \\
                \mathrm{s.t.}  &\displaystyle\sum\limits_{u\in C} w_u &\geq  z \quad \forall C \in \mathcal{C} \\
                    & \textbf{1}^Tw &=   1 \\
                    &   w &\geq   \textbf{0}
            \end{array}
        \end{equation*}
    \end{minipage}
    \begin{minipage}{.5\linewidth}
        \begin{equation*}
            \begin{array}{l@{\quad} r l l}
                (D)\\
                \min          &t   \\
                \mathrm{s.t.}  &\displaystyle\sum\limits_{v \in C \in \mathcal{C}} y_C &\leq  t \quad \forall v \in V \\
                    & \textbf{1}^Ty &=   1 \\
                    &   y &\geq   \textbf{0}
            \end{array}
        \end{equation*}
    \end{minipage}
    \newline
    \newline
    Observe that (i) holds if and only if the optimal value of (D) is greater than $\frac 12$ while (ii) holds if and only if the optimal value of (P) is greater than $\frac 12$.
    From strong duality we get the equivalence.
\end{proof}

In contrast to $Q_6$ we get the following consequence:
\begin{corollary}
    Deltas and blocker of extended odd holes are dense.
\end{corollary}

\begin{proof}
    As seen before, deltas and blockers of extended odd holes do not have a cover of size 1.
    For $\mathcal{C}=\Delta_n$ let $w=\left(\frac{n-2}{2n-3}, \frac 1{2n-3}, \ldots, \frac 1{2n-3}\right)$ and for blockers of extended odd holes let $w=\textbf{1}$.
\end{proof}

\chapter{The unifying theorem}
This chapter provides a polynomial characterisation whether a clutter has a restriction with a certain property.
This is a generalisation of Theorem 1.3 and Theorem 1.9 of \cite{restrictions}.
A common feature is the need of members whose union is the ground set in a clutter that is minimal in respect to the given property.
   \begin{definition}
       Let $\mathcal{S}$ be a set of clutters.
       A clutter $\mathcal{C} \in \mathcal{S}$ is called \emph{restriction-minimal}\index{restriction-minimal} in $\mathcal{S}$ if no proper restriction of $\mathcal{C}$ is in $\mathcal{S}$.
   \end{definition}

   \begin{definition}
       A set of clutters is called \emph{$k$-unifying}\index{$k$-unifying} if every restriction-minimal clutter in this set has $k$ members whose union is that clutters ground set.
   \end{definition}

   We can now formulate the main theorem.

   \begin{theorem}[unifying theorem]\label{unifying}
       Let $\mathcal{S}$ be a $k$-unifying set of clutters and $\mathcal{C}$ be a clutter over ground set $V$.
       Then the following two statements are equivalent:
       \leavevmode
       \begin{enumerate}[(i)]
           \item $\mathcal{C}$ has a restriction in $\mathcal{S}$,
           \item there are $k$ members $C_1, C_2, \ldots, C_k$ of $\mathcal{C}$ such that the clutter obtained by $\mathcal{C}$ after restricting $V - \bigcup_{i=1}^k C_i$ is a nontrivial clutter in $\mathcal{S}$.
       \end{enumerate}
   \end{theorem}

   \begin{proof}
       ($\impliedby$) is immediate.
       ($\implies$) Since clutters have finite ground set, $\mathcal{C}$ has a restriction-minimal restriction in $S$.
       Let this restriction be $\mathcal{C}\backslash I / J$.
       Since $\mathcal{S}$ is $k$-unifying we find $k$ members $C_1', C_2', \ldots, C_k'$ in this restriction, such that $\bigcup_{i=1}^k C_i' = V - (I \cup J)$.
       Let $C_i = C_i' \cup J$, which are members of $\mathcal{C}$.
       That yields $I=V-\bigcup_{i=1}^k C_i$, therefore such $k$ members are found.
   \end{proof}
   \begin{corollary}
       Let $\mathcal{C}$ be a clutter over ground set $V$ with $n$ elements and $m$ members.
       Let $\mathcal{S}$ be a $k$-unifying set of clutters for a fixed $k\geq 2$.
       Furthermore there is an oracle given which decides for a given clutter whether it is in $\mathcal{S}$ in polynomial time.
       Then one can decide in polynomial time dependent on $n$ and $m$ whether $\mathcal{C}$ has a nontrivial restriction in $\mathcal{S}$.
   \end{corollary}

   \begin{proof}
       For each set of not necessarily different $k$ members of $\mathcal{C}$, consider the restriction of the complement of the union of these members.
       If any of these restrictions restrictions is a nontrivial clutter in $\mathcal{S}$, there is a nontrivial restriction in $\mathcal{S}$ by \hyperref[unifying]{Theorem \ref*{unifying}}.
       Otherwise, there is no such restrictions.
       Since there are $m^k$ possibilities to choose the members, the restrictions can be calculated in polynomial time dependent on $n$ and $m$. The oracle decides whether the restriction is actually in $\mathcal{S}$, so this entire process can be done in polynomial time.
   \end{proof}

   In the third chapter, we will apply this to the set of intersecting clutters and in the fourth chapter to the set of dense clutters.
   We therefore get polynomial time algorithms to decide whether a clutter has an intersecting/dense restriction.

   \chapter{Intersecting restrictions}
   This chapter gives a simplified proof for Proposition 3.3 in \cite{restrictions}.
   The statement is also slightly generalized with the concept of $k$-wise intersection introduced in \cite{k-wise} because that does not change the proof.
   \begin{definition}
       A nontrivial clutter is called \emph{$k$-wise intersecting} if it has no cover of size 1 and every $k$ members (not necessarily different) have a common element.
   \end{definition}

   Note that the $2$-wise intersecting clutters are exactly the intersecting clutters.
   \begin{proposition}
       Let $k\geq 2$ be an integer.
       The set of $k$-wise intersecting clutters is $(k+1)$-unifying.
   \end{proposition}

   \begin{proof}
       Let $\mathcal{C}$ be $k$-wise intersecting such that no proper restriction is $k$-wise intersecting.
       We have to show that $\mathcal{C}$ has $(k+1)$ members whose union is the ground set.
       Choose $(k+1)$ members $C_1, C_2, \ldots, C_{k+1}$ of $\mathcal{C}$ such that $|\bigcap_{i=1}^{k+1} C_i|$ is minimal.
       We prove that the union of these members is the ground set $V$.
       Assume this is not the case.
       Then there is a $v \in V$ such that $v \not\in C_i$ for $i=1,2, \ldots, k+1$.
       Consider the restriction obtained by $\mathcal{C}$ after restricting $v$ (not a cover).
       Let this restriction be $\mathcal{C'}=\mathcal{C} \backslash v / J$.
       We get $J \subseteq \bigcap_{i=1}^{k+1} C_i$.

       Since $\mathcal{C'}$ is not $k$-wise intersecting and nonempty, we find $k$ members $C_1',\ldots, C_k'$ of $\mathcal{C'}$ with empty intersection.
       They imply $k$ members $C_1^*, C_2^*,\ldots, C_k^*$ in $\mathcal{C}$ with $\bigcap_{i=1}^k C_i^* \subseteq J$.

       Note that the intersection of these $k$ members is not empty since $\mathcal{C}$ is $k$-wise intersecting.
       We find an element $u \in \bigcap_{i=1}^k C_i^*$. Since $\{u\}$ is not a cover, we find $C_{k+1}^*$ with $u \not\in C_{k+1}^*$.
       We conclude
       \begin{align*}
           \bigcap_{i=1}^{k+1} C_i^* \subsetneq J \subseteq \bigcap_{i=1}^{k+1} C_i \;,
       \end{align*}
       a contradiction to the minimality assumption.
       Therefore, the union of these $(k+1)$ members is the ground set.


   \end{proof}

   For intersecting clutters we get the following consequence.
   \begin{corollary}
       The set of intersecting clutters is 3-unifying.
   \end{corollary}

   \begin{lemma}[\cite{restrictions}, Remark 1.2]
       A clutter has an intersecting restriction if and only if it has an intersecting minor.
   \end{lemma}

  \begin{proof}
      If a clutter has an intersecting restriction, this restriction is by definition a minor.

      Let $\mathcal{C}$ be clutter and $\mathcal{C} \backslash I /J$ be an intersecting minor.
      Let $\mathcal{C'}=\mathcal{C} \backslash I /J'$ be the minor obtained after restricting $I$.
      By the definition of a restriction and $\tau(\mathcal{C} \backslash I /J) \geq 2$, we get $J' \subseteq J$.If $\mathcal{C'}$ was trivial, so would $\mathcal{C} \backslash I / J$.
      If $\mathcal{C'}$ had two disjoint members, so would $\mathcal{C} \backslash I /J$ because it is not trivial.
      Since both is not the case, $\mathcal{C'}$ is intersecting.
  \end{proof}


   \chapter{Dense restrictions}
   In this chapter we will prove, that the set of dense clutters is 3-unifying.
   This is basically Proposition 4.5 of \cite{restrictions}.
   In contrast to the proof given there, we will not use deltas and blockers of extended odd hole minors directly.
   Instead, we will focus on the properties of a minimally dense clutter, a dense clutter such that no proper minor is dense.
   We will use similar ideas to the ones presented in the proof of Theorem 3 of \cite{deltas}, but the separation of dense clutters and delta or blocker of extended odd holes minors allows for some shortcuts.
   Furthermore we will get Theorem 3 of \cite{deltas} as an immediate consequence.
   Especially \hyperref[bipartite]{Lemma \ref*{bipartite}} provides more combinatorial insight to the connection of fractional packings of value 2 and covers of size 2.
   This feature was also exploited by Abdi, Cornuéjols and Superdock in Lemma 14 of \cite{lemma}.

   \begin{definition}
       Let $\mathcal{C}$ be a clutter over ground set $V$ with $\tau(\mathcal{C})=2$.
       The \emph{covering graph}\index{covering graph} of $\mathcal{C}$ is the graph with vertex set $V$ and the covers of size two of $\mathcal{C}$ as edges.
   \end{definition}
   Note that the covering graph of an extended odd hole is a cycle including all vertices and a clutter with a covering graph of a cycle including all vertices is an extended odd hole.
   The covering graph of a delta is a single vertice connected to all other vertices.

   \begin{lemma}\label{bipartite}
       Let $\mathcal{C}$ be a clutter over ground set $V$ with $\tau(\mathcal{C})=2$, connected covering graph and a fractional packing of value 2.
       Then the covering graph is bipartite and $\mathcal{C}$ has two members representing the colour classes, in particular there are members $L$ and $K$ of $\mathcal{C}$ with $K \cap L = \emptyset$ and $K \cup L = V$.
   \end{lemma}

   \begin{proof}
       For $C \in \mathcal{C}$ let $x_C$ be the value assigned to $C$ in the fractional packing of value 2.
       Let $B=\{b_1,b_2\}$ be an arbitrary cover of size 2 of $\mathcal{C}$ and $C$ be an arbitrary member of $\mathcal{C}$ with $x_{C} > 0$.
       We first prove $|B\cap C| = 1$.
       Since $B$ is a cover, we have $|B\cap C| \geq 1$.
       Assume $|B \cap C| > 1$, so $B \subseteq C$.
       Let $B_1$ be the set of members of $\mathcal{C}$ containing $b_1$ and $B_2$ be the set of members containing $b_2$.
       We conclude
       \begin{align*}
           2 \geq \sum_{C'\in B_1} x_{C'} + \sum_{C' \in B_2} x_{C'} = \sum_{C' \in B_1 \text{ or } C' \in B_2} x_{C'} + \sum_{C' \in B_1 \text{ and } C' \in B_2} x_{C'} \geq 2 + x_C > 2 \;,
       \end{align*}
       a contradiction. Therefore $|B\cap C| = 1$.

       Since the cover of size 2 was arbitrary, each member $C$ with $x_C>0$ has exactly one element with each of these covers in common.
       In the covering graph, such a member is a stable set and vertex cover.
       Given whether an element $s \in V$ is contained in $C$ or not uniquely determines whether an element $t \in V$ has to be contained in $C$ since connectivity of the covering graph assures an $s$-$t$-path which has to be alternating.
       In particular, the covering graph cannot contain an odd cycle, since that would result in two paths between the same vertices of different parity in length. Thus, the covering graph is bipartite and we get two colour classes.

       Since paths between vertices of the same colour class have even length and paths between vertices of different colour classes have odd length, a member which contains one element of a colour class, exactly consists of the elements of this colour class.
       It is impossible, that all members with $x_C > 0$ are only one of the two colour classes because that would be the only member in the fractional packing and a value of 2 would not be possible.
       Therefore each colour class is represented by a member, so there are members $K$ and $L$ with $K \cap L = \emptyset$ and $K \cup L = V$.

   \end{proof}

   \begin{definition}
       A clutter is called \emph{minimally dense} if it is dense and no proper minor is dense, i.e. each proper minor withcovering number at least 2 has a fractional packing of value 2.
       A clutter is called \emph{strictly dense} if it is dense and no proper restriction is dense.
   \end{definition}
   \begin{lemma}\label{connectivity}
       Let $\mathcal{C}$ be a minimally dense clutter.
       Then $\tau(\mathcal{C})=2$ and the covering graph of $C$ is connected.
   \end{lemma}

   \begin{proof}
       If an element $v \in V$ does not appear in a cover of size two of $\mathcal{C}$, the proper minor $\mathcal{C} \backslash v$ has covering number at least 2.
       Thus, this minor has a fractional packing of value 2 which is also a fractional packing of value 2 for $\mathcal{C}$, a contradiction.
       Therefore $\tau(\mathcal{C}) = 2$ and each element of the ground set appears in a cover of size 2.
       Let $G$ be the covering graph of $\mathcal{C}$.

       Assume $G$ is not connected.
       Let $A$ be the vertex set of a component of $G$ and $B = V - A$.
       Let $H$ be the subgraph of $G$ induced by $A$.
       Note that $A$ is a cover of $\mathcal{C}$ since it contains at least one cover.

       Consider the minor $\mathcal{C'}=\mathcal{C}/B$.
       If $\mathcal{C'}$ has a cover of size 1, this would also be a cover of $\mathcal{C}$, since no member is entirely contracted as $A$ is a cover.
       Hence, $\mathcal{C'}$ has a fractional packing of value 2.
       The covering graph $H'$ of $\mathcal{C'}$ contains the edges of $H$ and is therefore connected.
       By applying \hyperref[bipartite]{Lemma \ref*{bipartite}} on $\mathcal{C'}$ we get that $H'$ is bipartite and the two colour classes $K$ and $L$ are members in $\mathcal{C'}$, implying that $K$ and $L$ are not covers in $\mathcal{C'}$ and $\mathcal{C}$.

       If both of the minors $\mathcal{C} \backslash K / L$ and $\mathcal{C} \backslash L / K$ have a fractional packing of value 2, we can add these two fractional packings in the canonical way and divide by 2.
       We get a fractional packing of value 2 for $\mathcal{C}$ which is not possible.
       Hence, one of these minors has a cover of size 1.
       Let this cover be $\{b\}$.
       The proper minor $\mathcal{C''}=\mathcal{C} \backslash b / (B-\{b\})$ has no cover of size 1 since there is no edge between the vertex sets $A$ and $B$.
       Using the same argument as for $\mathcal{C'}$, the covering graph of $\mathcal{C''}$ contains the edges of $H$, is bipartite and has the colour classes as members.
       Therefore $K$ and $L$ are members of $\mathcal{C''}$, but not covers.
       That yields that neither $K \cup \{b\}$ nor $L \cup \{b\}$ are covers of $\mathcal{C}$, a contradiction.
       Hence, $G$ is connected.
   \end{proof}


   \begin{proposition}\label{twomember}
       Let $\mathcal{C}$ be a minimally dense clutter over ground set $V$.
       Then $\mathcal{C}$ is a delta or the blocker of an extended odd hole.
   \end{proposition}

   \begin{proof}
       By the previous lemma, the covering graph $G$ of $\mathcal{C}$ is connected.
       If $G$ contains a cycle, contrating all other elements leads to a minor with covering number at least 2, but the covering graph is not bipartite.
       If the minor has a fractional packing of value 2, \hyperref[bipartite]{Lemma \ref*{bipartite}} implies a contradiction. Therefore the minor is dense and thus not proper.
       In conclusion, $G$ is a cycle and does not contain further cycles, i.e. no additonal chords.
       Therefore $\mathcal{C}$ is the blocker of an extended odd hole or a delta if $|V|=3$.

       We can now assume that $G$ is bipartite.
       Let $X$ and $Y$ be the colour classes of $G$.
       Assume that $|X|>1$ and $|Y|>1$.
       We will show that $\mathcal{C}$ is a delta or find two disjoint members.
       Remove an $x \in X$ from $G$ such that the number of vertices from $X$ in the same component is maximal.
       Let this maximal component be $M_x$.
       If this number is not $|X|-1$, take an $x' \in X$ that is not in $M_x$.
       Removing $x'$ instead of $x$ does not disconnect $M_x$. Furthermore $x$ is connected to $M_x$, a contradiction to the maximality.
       Therefore there is an $x \in X$ such that removing $x$ from $G$ does not disconnect the other vertices in $X$.

       Let $G'$ be the graph resulting from $G$ after removing $x$.
       The components of $G'$ consist of one component containing all other vertices in $X$ and some vertices from $Y$.
       All other components have to be a single vertex from $Y$, because there are no edges between vertices in $Y$.
       Let this set of isolated vertices be $Z$.
       We get that $\{x,z\}$ is an edge in $G$ for all $z \in Z$ since $G$ is connected.

       Assume there is a $y \in Y$ such that $y \cup Z$ is a cover of $\mathcal{C}$ ($y \in Z$ is allowed).
       Let $Z' \subseteq \{y\} \cup Z$ be a minimal cover.
       Note that $|Z'| > 2$ because of the covering graph.
       If $\{x,y\} \in E(G)$, consider the minor $\mathcal{C} /(V - (\{x\} \cup Z'))$.
       In this minor, $Z'$ is still a minimal cover.
       Additionally $\{x,z\}$ is a minimal cover for all $z \in Z'$.
       Therefore this minor is a delta.
       Deltas are dense clutters, therefore $\mathcal{C}$ is a dense clutter in this case.

       If $\{x,y\} \not\in E(G)$, consider the minor $\mathcal{C} \backslash y / (V - (\{x\} \cup (Z'-\{y\})))$ instead.
       Then $Z' - \{y\}$ is a minimal cover of this minor and we can also conclude that this minor is a delta ($|Z'-\{y\}| \geq 2$).

       We can now assume that there is no such $y \in Y$ such that $y \cup Z$ is a cover.
       Starting with the minor $\mathcal{C} / x$ we delete elements of $Z$ one by one if they are still isolated vertices of the covering graph in the resulting minor.
       Observe that we never delete a cover since $Z$ is not a cover.
       Let $Z^* \subseteq Z$ be the vertices that are deleted at the end of this process.
       The vertices of $Z - Z^*$ can only be connected to vertices in $X$ due to the assumption.
       The minor $\mathcal{C} \backslash Z^* / x$ has no cover of size 1 because only isolated vertices were deleted.
       By the choice of $x$ and $Z^*$ the covering graph of this minor is connected and by $|X|>1$ the minor is not the empty set.
       By \hyperref[bipartite]{Lemma \ref*{bipartite}} this minor has a member $K \subseteq X$.
       In the original clutter $\mathcal{C}$ this member is also a subset of $X$.
       Using the same argument for $Y$, we get that $\mathcal{C}$ is a delta or a member which is a subset of $Y$.
       These two members are disjoint, which is not possible.
       The remaining case is $|X|=1$ or $|Y|=1$. WLOG $|X|=1$.
       Then the covering graph is a single vertex $x$ connected to all other vertices.
       Considering the minors $\mathcal{C} / \{y\}$ for all $y \in Y$ and applying \hyperref[bipartite]{Lemma \ref*{bipartite}}, $\{x\}$ is an element in these minors.
       But $\{x\}$ is not a member of $\mathcal{C}$ because there would be disjoint members or a cover of size 1.
       Hence $\{x,y\}$ is a member of $\mathcal{C}$ for all $y \in Y$.
       Furthermore $Y$ is a cover of $\mathcal{C}$ and due to these members, it is also minimal.
       We conclude that $\mathcal{C}$ is indeed a delta.
   \end{proof}
   \begin{corollary}[\cite{deltas}, Theorem 3]
       Let $\mathcal{C}$ be a dense clutter over ground set $V$ such that $\tau(\mathcal{C}) \geq 2$.
       Then $\mathcal{C}$ has a delta or blocker of extended odd hole minor.
   \end{corollary}

   \begin{proof}
       This is immediate from the fact, that each dense clutter has a minimally dense minor.
   \end{proof}

   \begin{corollary}\label{threemember}
       Let $\mathcal{C}$ be a minimally dense clutter over ground set $V$.
       Then $\mathcal{C}$ has three members $C_1, C_2$ and $C_3$ with empty intersection whose union is the ground set.
   \end{corollary}

   \begin{proof}
       We will prove the stronger result, that such a clutter has two members $C_1$ and $C_2$ whose union is the ground set and $|C_1 \cap C_2|=1$.
       We can then choose an arbitrary $C_3$ not containing the common element which completes the proof.
       If $\mathcal{C}$ is a delta, choose $C_1=\{1,2\}$ and $C_2=\{2,3,\ldots,n\}$.

       If $\mathcal{C}$ is the blocker of an extended odd hole, take an arbitrary $v \in V$ and consider the minor $\mathcal{C} / \{v\}$.
       The covering graph of this minor is connected and bipartite since it is a path.
       By \hyperref[bipartite]{Lemma \ref*{bipartite}} we find two disjoint members $K$ and $L$ whose union is $V-\{v\}$.
       They both contain $v$ in $\mathcal{C}$ since $\mathcal{C}$ has no disjoint members.
   \end{proof}

   This corollary also shows, that a dense clutter has a delta or a blocker of an extended odd hole minor with the additonal property that for each $v \in V$ we find two members intersecting at $v$ whose union is $V$.

   The following Lemma is a useful tool to bridge the gap from minimally dense to strictly dense clutters.
   \begin{lemma}\label{covers}
       Let $\mathcal{C}$ be a strictly dense clutter over ground set $V$ and $J \subseteq V$ such that $\mathcal{C} / J$ is dense.
       Then for each $v \in J$ there is a $w \in V-J$ such that $\{v,w\}$ is a cover of size 2 in $\mathcal{C}$.
   \end{lemma}

   \begin{proof}
       Assume there is a $v \in J$ such that no such cover of size 2 exists.
       Then the minor $\mathcal{C} \backslash \{v\} / (J-\{v\})$ has no cover of size 1 and therefore covering number at least 2.
       Furthermore this minor has no fractional packing of value 2 since it contains only a subset of the members in $\mathcal{C} / J$.
       Therefore this minor is dense.
       The restriction obtained from $\mathcal{C}$ after restricting $v$ contracts a subset of $J-\{v\}$ since $\tau(\mathcal{C} \backslash \{v\} / (J-\{v\}))\geq 2$.
       This implies that that restriction is also dense, a contradiction.
   \end{proof}

   We are now ready to prove the main proposition of this chapter.
   \begin{proposition}[\cite{restrictions}, Proposition 4.5]
       The set of dense clutters is 3-unifying.
   \end{proposition}

   \begin{proof}
       Let $\mathcal{C}$ be a strictly dense clutter over ground set $V$.
       Choose $U$ such that $\mathcal{C} / U$ is dense but no proper contraction minor is.
       Let $\mathcal{C} \backslash I / (U \cup U')$ be a proper minor of $\mathcal{C} /U$ with covering number at least 2.
       If $I \neq \emptyset$, the restriction $\mathcal{C} \backslash I / J$ is not dense and therefore has a fractional packing of value 2.
       Since $J \subseteq (U \cup U')$, the minor $\mathcal{C} \backslash I / (U \cup U')$ also has a fractional packing of value 2.
       If $I=\emptyset$ we get the same result by the definition of $U$.

       Hence, $\mathcal{C}/U$ is minimally dense.
       By \hyperref[threemember]{Corollary \ref*{threemember}} we find three members $C_1', C_2'$ and $C_3'$ in $\mathcal{C}/U$ with empty intersection and union $V - U$.

       Let $C_1, C_2, C_3 \in \mathcal{C}$ such that $C_i' \subseteq C_i \subseteq C_i' \cup U$ for $i=1,2,3$.
       We will proof $C_1 \cup C_2 \cup C_3 = V$.
       Suppose there is a $v \in V-(C_1 \cup C_2 \cup C_3)$.
       Clearly, $v \in U$.
       By \hyperref[covers]{Lemma \ref*{covers}} there is a cover $\{v,w\}$ with $w \in V-U$.
       Since none of the three members contains $U$, they all contain $w$, a contradiction.
   \end{proof}

   \printbibliography[title={References}]
   \printindex
\end{document}
