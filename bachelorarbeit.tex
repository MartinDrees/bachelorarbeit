\documentclass[a4paper, 12pt, twoside=false]{scrbook}

\usepackage[utf8]{inputenc}
\usepackage[english]{babel}
\usepackage{amsthm}
\usepackage{amsmath}
\usepackage{amssymb}
\usepackage{enumerate}
\usepackage[colorlinks=true]{hyperref}

\usepackage[backend=biber]{biblatex}
\addbibresource{sample.bib}

\theoremstyle{definition}
\newtheorem*{definition}{Definition}
\newtheorem{theorem}{Theorem}[chapter]
\newtheorem{corollary}[theorem]{Corollary}
\newtheorem{lemma}[theorem]{Lemma}
\newtheorem{proposition}[theorem]{Proposition}

\newcommand*{\id}{\ensuremath{\mathrm{id}}}
\newcommand*{\IC}{\ensuremath{\mathbb{C}}}
\newcommand*{\IR}{\ensuremath{\mathbb{R}}}
\newcommand*{\IQ}{\ensuremath{\mathbb{Q}}}
\newcommand*{\IZ}{\ensuremath{\mathbb{Z}}}
\newcommand*{\IN}{\ensuremath{\mathbb{N}}}
\let\eps\varepsilon

\title{Intersecting clutters}
\author{Martin Drees}
\begin{document}
   \maketitle
   \tableofcontents
   \chapter{Introduction}
   \begin{definition}
       A \emph{clutter} over ground set $V$ is a family of subsets of $V$ called members such that no member contain another.
       The clutters $\{\}$ and $\{\emptyset\}$ are called \emph{trivial} clutters.
   \end{definition}

\section{Covers and packings}
\begin{definition}
    Let $\mathcal{C}$ be a clutter over ground set $V$.
    A \emph{cover} of $\mathcal{C}$ is a set $B \subseteq V$ such that $B \cap C \neq \emptyset$ for all members $C \in \mathcal{C}$.
    The \emph{covering number} $\tau(\mathcal{C})$ is the minimum cardinality of a cover.
\end{definition}

\begin{definition}
    A \emph{packing} of a clutter is a set of pairwise disjoint members.
    The \emph{packing number} $\nu(\mathcal{C})$ is the maximum cardinality of a packing.
\end{definition}

For a nontrivial clutter $\mathcal{C}$ we have $\tau(\mathcal{C}) \geq \nu(\mathcal{C})$ since each of the pairwise disjoint members in a packing contains one element of a cover.

\section{Minors and restrictions}
\begin{definition}
    Let $\mathcal{C}$ be a clutter over ground set $V$ and $I, J \subseteq V$ disjoint subsets.
    The \emph{minor} of $\mathcal{C}$ obtained after \emph{deleting} $I$ and \emph{contracting} $J$ is the clutter $\mathcal{C} \backslash I / J$ over ground set $V - (I \cup J)$ whose members are the inclusion-wise minimal sets of $\{C-J : C~\in~\mathcal{C},\, C~\cap~I = \emptyset\}$.
    If $I \cup J \neq \emptyset$ the minor is called \emph{proper}.
\end{definition}

\begin{definition}
    Let $\mathcal{C}$ be a clutter over ground set $V$ and $I \subseteq V$.
    Let
    \begin{align*}
        J := \{u \in V-I : \{u\} \text{ is a cover of } C\backslash I\} \;.
    \end{align*}
    The minor $\mathcal{C} \backslash I / J$ is the \emph{restriction} of $\mathcal{C}$ after restricting $I$.
\end{definition}

By the definition of the set $J$, a restriction is either trivial or $\tau(\mathcal{C}) \geq 2$ since a cover of size 1 would have been contracted.
\section{Intersecting clutters}
\begin{definition}
    A clutter is \emph{intersecting} if $\tau(\mathcal{C}) > \nu(\mathcal{C}) = 1$, i.e. every two members intersect, but not all members have a common element.
\end{definition}

\section{Dense clutters}
\begin{definition}
    Let $\mathcal{C}$ be a clutter over ground set $V$.
    A \emph{fractional packing} is a vector $y \in \IR_{\geq0}^{\mathcal{C}}$ such that
    \begin{align*}
        \sum_{v \in C \in \mathcal{C}} y_C \leq 1 \quad \forall v \in V \,.
    \end{align*}
    The \emph{value} of the fractional packing is $\textbf{1}^Ty$.
\end{definition}
\begin{definition}
    A clutter $\mathcal{C}$ is called \emph{dense}, if $\tau(\mathcal{C}) \geq 2$ and $\mathcal{C}$ has no fractional packing of value 2.
\end{definition}

Observe that a dense clutter is also intersecting since two disjoint members would yield a fractional packing of value 2.
The converse is not necessaraly true.
The clutter $Q_6$ of triangles in the $K_4$ has a fractional packing of value 2 by assigning $\frac 12$ to each of the four triangles, but is intersecting.

The following result gives a certificate for a dense clutter.
\begin{lemma}
    Let $\mathcal{C}$ be a clutter with $\tau(\mathcal{C})\geq 2$ over ground set $V$.
    Then the following are equivalent:
    \leavevmode
    \begin{enumerate}[(i)]
        \item $\mathcal{C}$ is dense,
        \item there is a $w \in \IR_{\geq 0}^V$ with $\textbf{1}^Tw=1$ such that $\sum_{u \in C} w_u > \frac 12$ for all C $\in \mathcal{C}$.
    \end{enumerate}
\end{lemma}

\begin{proof}
    Consider the following dual pair of linear programs
    \newline
    \newline
    \begin{minipage}{.5\linewidth}
        \begin{equation*}
            \begin{array}{l@{\quad} r l l}
                (P)\\
                \max          &z   \\
                \mathrm{s.t.}  &\displaystyle\sum\limits_{u\in C} w_u &\geq  z \quad \forall C \in \mathcal{C} \\
                    & \textbf{1}^Tw &=   1 \\
                    &   w &\geq   \textbf{0}
            \end{array}
        \end{equation*}
    \end{minipage}
    \begin{minipage}{.5\linewidth}
        \begin{equation*}
            \begin{array}{l@{\quad} r l l}
                (D)\\
                \min          &t   \\
                \mathrm{s.t.}  &\displaystyle\sum\limits_{v \in C \in \mathcal{C}} y_C &\leq  t \quad \forall v \in V \\
                    & \textbf{1}^Ty &=   1 \\
                    &   y &\geq   \textbf{0}
            \end{array}
        \end{equation*}
    \end{minipage}
    \newline
    \newline
    Observe that (i) holds if and only if the optimal value of (D) is greater than $\frac 12$ while (ii) holds if and only if the optimal value of (P) is greater than $\frac 12$.
    From strong duality we get the equivalence.
\end{proof}

In contrast to $Q_6$ we get the following consequnece:
\begin{corollary}
    Deltas and blocker of extended odd holes are dense.
\end{corollary}

\begin{proof}
    As seen before, deltas and blockers of extended odd holes do not have a cover of size 1.
    For $\mathcal{C}=\Delta_n$ let $w=\left(\frac{n-2}{2n-3}, \frac 1{2n-3}, \ldots, \frac 1{2n-3}\right)$ and for blockers of extended odd holes let $w=\textbf{1}$.
\end{proof}

\section{The unifying theorem}
   \begin{definition}
       Let $\mathcal{S}$ be a set of clutters.
       A clutter $\mathcal{C} \in \mathcal{S}$ is called \emph{restriction-minimal} in $\mathcal{S}$ if no proper restriction of $\mathcal{C}$ is in $\mathcal{S}$.
   \end{definition}

   \begin{definition}
       A set of clutters is called \emph{$k$-unifying} if every restriction-minimal Clutter in this set has $k$ members whose union is the ground set.
   \end{definition}

   \begin{theorem}
       Let $\mathcal{S}$ be a $k$-unifying set of clutters and $\mathcal{C}$ be a clutter over ground set $V$.
       Then the following two statements are equivalent:
       \leavevmode
       \begin{enumerate}[i)]
           \item $\mathcal{C}$ has a restriction in $\mathcal{S}$.
           \item There are $k$ members $C_1, C_2, \ldots, C_k$ of $\mathcal{C}$ such that the clutter obtained by $\mathcal{C}$ after restricting $V - \bigcup_{i=1}^k C_i$ is in $\mathcal{S}$.
       \end{enumerate}
   \end{theorem}

   \begin{proof}
       ($\impliedby$) is immediate.
       ($\implies$) Since clutters have finite ground set, $\mathcal{C}$ has a restriction-minimal restriction.
       Let this restriction be $\mathcal{C}\backslash I / J$.
       Since $\mathcal{S}$ is $k$-unifying, we find $k$ members $C_1', C_2', \ldots, C_k'$ in this restriction, such that $\bigcup_{i=1}^k C_i' = V - (I \cup J)$.
 Let $C_i = C_i' \cup J$.
       That yields $I=V-\bigcup_{i=1}^k C_i$, therefore such $k$ members are found.
   \end{proof}
   \begin{corollary}
       Let $\mathcal{C}$ be a clutter and $\mathcal{S}$ be a set of clutters.
       Furthermore there is an oracle given which decides for a given clutter whether it is in $\mathcal{S}$.
       Then one can decide in polynomial time whether $\mathcal{C}$ has a restriction in $\mathcal{S}$.
   \end{corollary}

   In the second chapter, we will apply this to the set of intersecting clutters and in the third chapter to the set of dense clutters.
   We therefore get polynomial time algorithms to decide whether a clutter has an intersecting/dense restriction.

   \chapter{Intersecting restrictions}
   \begin{proposition}
       Let $k\geq 2$ be an integer.
       The set of $k$-wise intersecting clutters is $(k+1)$-unifying.
   \end{proposition}

   \begin{proof}
       Let $\mathcal{C}$ be $k$-wise intersecting such that no proper restrictionis $k$-wise intersecting.
       We have to show that $\mathcal{C}$ has $(k+1)$ members whose union is the ground set.
       Choose $(k+1)$ members $C_1, C_2, \ldots, C_{k+1}$ of $\mathcal{C}$ such that $|\bigcap_{i=1}^{k+1} C_i|$ is minimal.
       We prove that the union of these members is the ground set $V$.
       Assume this is not the case.
       Then there is a $v \in V$ such that $v \not\in C_i$ for $i=1,2, \ldots, k+1$.
       Consider the restriction obtained by $\mathcal{C}$ after restricting $v$.
       Let this restriction be $\mathcal{C'}=\mathcal{C} \backslash v / J$.
       Note that by the definition of a restriction, we get $J \subseteq \bigcap_{i=1}^{k+1} C_i$.

       If this restriction is not a proper restriction, $\mathcal{C'}$ is trivial.
       That means $\mathcal{C'}=\{\}$ or $\mathcal{C'}=\{\emptyset\}$.
       The first case is not possible because $\{v\}$ would be a cover of $\mathcal{C}$.
       In the second case, there is a member $C^* \in \mathcal{C}$ with $C^* \subseteq J$.

       If the restriction is a proper restriction, $\mathcal{C'}$ is not $k$-wise intersecting. We therefore find $k$ members $C_1',\ldots, C_k'$ of $\mathcal{C'}$ with empty intersection. We get $k$ members $C_1^*, C_2^*,\ldots, C_k^*$ in $\mathcal{C}$ with $\bigcap_{i=1}^k C_i^* \subseteq J$.
       In the first case we take $k-1$ arbitrary further members to get the same result.

       Note that the intersection of these $k$ members is not empty since $\mathcal{C}$ is $k$-wise intersecting.
       We find an element $u \in \bigcap_{i=1}^k C_i^*$. Since $\{u\}$ is not a cover, we find $C_{k+1}^*$ with $u \not\in C_{k+1}^*$.
       We conclude
       \begin{align*}
           \bigcap_{i=1}^{k+1} C_i^* \subsetneq J \subseteq \bigcap_{i=1}^{k+1} C_i \;,
       \end{align*}
       a contratiction to the minimality assumption of these $(k+1)$ members.
       Therefore, the union of these $(k+1)$ members is the ground set.


   \end{proof}

   For intersecting clutters we get the following consequence.
   \begin{corollary}
       The set of intersecting clutters is 3-unifying.
   \end{corollary}

   \begin{lemma}
       A clutter has an intersecting restriction if and only if it has an intersecting minor.
   \end{lemma}

  \begin{proof}
      If a clutter has an intersecting restriction, this restriction is by definition a minor.

      Let $\mathcal{C}$ be clutter and $\mathcal{C} \backslash I /J$ be an intersecting minor.
      Let $\mathcal{C'}=\mathcal{C} \backslash I /J'$ be the minor obtained after restricting $I$.
      By the definition of a restriction and $\tau(\mathcal{C} \backslash I /J) \geq 2$, we get $J' \subseteq J$.If $\mathcal{C'}$ was trivial, so would $\mathcal{C} \backslash I / J$.
      If $\mathcal{C'}$ had two disjoint members, so would $\mathcal{C} \backslash I /J$ because it is not trivial.
      Since both is not the case, $\mathcal{C'}$ is intersecting.
  \end{proof}


   \chapter{Dense restrictions}
   In this chapter we will prove, that the set of dense clutters is 3-unifying.
   \begin{definition}
       Let $\mathcal{C}$ be a clutter over ground set $V$ with $\tau(\mathcal{C})=2$.
       The \emph{covering graph} of $\mathcal{C}$ with vertex set $V$ and the covers of size two of $\mathcal{C}$ as edges.
   \end{definition}

   \begin{lemma}\label{bipartite}
       Let $\mathcal{C}$ be a clutter over ground set $V$ with $\tau(\mathcal{C})=2$, connected covering graph and a fractional packing of value 2.
       Then the covering graph is bipartite and $\mathcal{C}$ has two members representing the colour classes, in particular there are member $L$ and $K$ of $\mathcal{C}$ with $K \cap L = \emptyset$ and $K \cup L = V$.
   \end{lemma}

   \begin{proof}
       For $C \in \mathcal{C}$ let $x_C$ be the value assigned to $C$ in the fractional packing of value 2.
       Let $B=\{b_1,b_2\}$ be an arbitrary cover of size 2 of $\mathcal{C}$ and $C$ be an arbitrary member of $\mathcal{C}$ with $x_{C} > 0$.
       We first prove $|B\cap C| = 1$.
       Since $B$ is a cover, we have $|B\cap C| \geq 1$.
       Assume $|B \cap C| > 1$, so $B \subseteq C$.
       We conclude
       \begin{align*}
           2 \geq \sum_{b_1 \in C_i} x_{C_i} + \sum_{b_2 \in C_i} x_{C_i} = \sum_{b_1 \in C_i \text{ or } b_2 \in C_i} x_{C_i} + \sum_{b_1 \in C_i \text{ and } b_2 \in C_i} x_{C_i} \geq 2 + x_C > 2 \;,
       \end{align*}
       a contradiction. Therefore $|B\cap C| = 1$.

       Since the cover of size 2 was arbitrary, each member $C$ with $x_C$ has exactly one element with each of these covers in common.
       In the covering graph, such a member is a stable set and vertex cover.
       Given whether an element $s \in V$ is contained in $C$ or not uniquly determines whether an element $t \in V$ has to be contained in $C$ since connectivity of the covering graph assures a $s-t$-path which has to be alternating.
       In particular, the covering graph cannot contain an odd cycle, since that would result in two paths between the same vertices of different parity in length. Thus, the covering graph is bipartite and we get two colour classes.

       Since paths between vertices of the same colour class have even length and paths between vertices of different colour classes have odd length, a member which contains one element of a colour class, exactly consists of the elements of this colour class.
       It is impossible, that all members with $x_C > 0$ are only one of the two colour classes because that would be the only member in the fractional packing and a value of 2 would not be possible.
       Therefore each colour class is represented by a member, so there are members $K$ and $L$ with $K \cap L = \emptyset$ and $K \cup L = V$.

   \end{proof}

   \begin{proposition}\label{twomember}
       Let $\mathcal{C}$ be a dense clutter over ground set $V$ such that every proper minor with covering number at least 2 has a fractional packing.
       Then $\mathcal{C}$ has two members $C_1$ and $C_2$ such that $|C_1 \cap C_2|=1$ and $C_1 \cup C_2 = V$.
   \end{proposition}

   \begin{proof}
       If an element $v \in V$ does not appear in a cover of size two of $\mathcal{C}$, the proper minor $\mathcal{C} \backslash v$ has covering number at least 2.
       Thus, this minor has a fractional packing of value 2 which is also a fractional packing of value 2 for $\mathcal{C}$, a contradiction.
       Therefore $\tau(\mathcal{C}) = 2$ and each element of the ground set appears in a cover of size 2.
       Let $G$ be the covering graph of $\mathcal{C}$.
       We will first proof, that $G$ is connected.
       Assume this is not the case.
       Let $A$ be the vertex set of a component of $G$ and $B = V - A$.
       Let $H$ be the subgraph of $G$ induced by $A$.
       Note that $A$ is a cover of $\mathcal{C}$ since it contains at least one cover.

       Consider the minor $\mathcal{C'}=\mathcal{C}/B$.
       If $\mathcal{C'}$ has a cover of size 1, this would also be a cover of $\mathcal{C}$, since no member is entirely contracted.
       Hence, $\mathcal{C'}$ has a fractional packing of value 2.
       Furthermore, every cover of $\mathcal{C}$ disjoint to $B$ is a cover of $\mathcal{C'}$.
       Therefore the covering graph of $\mathcal{C'}$ contains the edges of $H$ and is therefore connected.
       Applying \hyperref[bipartite]{Lemma \ref*{bipartite}} to $\mathcal{C'}$ yields, that the covering graph is bipartite, implying that $H$ is bipartite.
       Let $K$ and $L$ be the two sets representing the colour classes of $H$.
       We also get, that the two colour classes are members in $\mathcal{C'}$, implying that $K$ and $L$ are not covers in $\mathcal{C}$.

       If both of the minors $\mathcal{C} \backslash K / L$ and $\mathcal{C} \backslash L / K$ have a fractional packing of value 2, we can add these two fractional packings in the canonical way and divide by 2.
       We get a fractional packing of value 2 for $\mathcal{C}$ which is not possible.
       Hence, one of these minors has a cover of size 1.
       Let this cover be $\{b\}$.
       The proper minor $\mathcal{C''}=\mathcal{C} \backslash b / (B-\{b\}$ has no cover of size 1 since there is no edge between the vertex sets $A$ and $B$.
       Using the same argument as for $\mathcal{C'}$, the covering graph of $\mathcal{C''}$ contains the edges of $H$, is bipartite and has the colour classes as members.
       Therefore $K$ and $L$ are members of $\mathcal{C''}$, but not covers.
       That yields that neither $K \cup \{b\}$ nor $L \cup \{b\}$ are covers of $\mathcal{C}$, a contradiction.
       Hence, $G$ is connected.

       Consider a leaf $s \in V$ of an arbitrary spanning tree of $G$.
       If the minor $\mathcal{C} / s$ has a cover of size 1, $V - s$ is connected but contains no edge.
       That implies $|V| \leq 2$, but then $\mathcal{C}$ cannot be dense.
       Otherwise $\mathcal{C} / s$ has a fractional packing of value 2 and connected covering graph.
       Therefore, we find two disjoint members $K$ and $L$ representing the colour classes whose union is $V-s$.
       In $\mathcal{C}$ both members have to contain $s$ to avoid a fractional packing of value 2.
       So $C_1=K\cup \{b\}$ and $C_2=L\cup \{b\}$ fulfill the condition.
   \end{proof}

   We are now ready to prove the main proposition of this chapter.
   \begin{proposition}
       The set of dense clutters is 3-unifying.
   \end{proposition}

   \begin{proof}
       Let $\mathcal{C}$ be a dense clutter over ground set $V$ such that every proper restriction of $\mathcal{C}$ is not dense.
       Choose $U$ such that $\mathcal{C} / U$ is dense but no proper contraction minor is.
       Let $\mathcal{C} \backslash I / (U \cup U')$ be a proper minor of $\mathcal{C} /U$ with covering number at least 2.
       If $I \neq \emptyset$, the restriction $\mathcal{C} \backslash I / J$ is not dense and therefore has a fractional packing of value 2.
       Since $J \subseteq (U \cup U')$, the minor $\mathcal{C} \backslash I / (U \cup U'$ also has a fractional packing of value 2.
       If $I=\emptyset$ we get the same result by the definition of $U$.

       Hence, $\mathcal{C}/U$ fulfills the condition of \hyperref[twomember]{Proposition \ref*{twomember}}.
       We find two members $C_1'$ and $C_2'$ with $|C_1' \cap C_2'| = 1$ and $C_1' \cup C_2' = V - U$.
       Choose $C_3'$ which does not contain the element of the intersection.
       This is possible since there is no cover of size 1.

       Let $C_1, C_2, C_3 \in \mathcal{C}$ such that $C_i' \subseteq C_i \subseteq C_i' \cup U$ for $i=1,2,3$.
       We will proof $C_1 \cup C_2 \cup C_3 = V$.
       Suppose there is $v \in V-(C_1 \cup C_2 \cup C_3)$.
       Clearly, $v \in U$.
       The minor $\mathcal{C'} = \mathcal{C} \backslash v / (U-\{v\})$ contains a subset of the members of $\mathcal{C}/U$.
       Therefore $\mathcal{C'}$ has no fractional packing.
       Since $C_1', C_2', C_3' \in \mathcal{C'}$ and they have empty intersection, $\mathcal{C'}$ has no cover of size 1.
       That implies $\mathcal{C'}$ is dense.
       So the proper restriction $\mathcal{C}\backslash v / J$ is dense, a contradiction.
       Therefore $\mathcal{C}$ has three members whose union is the ground set and the set of dense clutters is 3-unifying.
   \end{proof}

   \nocite{*}
   \printbibliography[title={References}]
\end{document}
