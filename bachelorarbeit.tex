\documentclass[a4paper, 12pt, twoside=false]{scrbook}

\usepackage[utf8]{inputenc}
\usepackage[english]{babel}
\usepackage{amsthm}
\usepackage{amsmath}
\usepackage{amssymb}
\usepackage{enumerate}

\theoremstyle{definition}
\newtheorem{definition}{Definition}
\newtheorem{theorem}{Theorem}[section]
\newtheorem{corollary}{Corollary}
\newtheorem{lemma}{Lemma}
\newtheorem{proposition}{Proposition}

\title{Intersecting Clutters}
\author{Martin Drees}
\begin{document}
   \maketitle
   \tableofcontents
   \chapter{Introduction}
   \begin{definition}
       Let $\mathcal{S}$ be a set of Clutters.
       A Clutter $\mathcal{C} \in \mathcal{S}$ is called \emph{restriction-minimal} in $\mathcal{S}$ if no proper restriction of $\mathcal{C}$ is in $\mathcal{S}$.
   \end{definition}

   \begin{definition}
       A set of Clutters is called \emph{$k$-unifying} if every restriction-minimal Clutter in this set has $k$ members whose union is the ground set.
   \end{definition}

   \begin{theorem}
       Let $\mathcal{S}$ be a $k$-unifying set of clutters and $\mathcal{C}$ be a Clutter over ground set $V$.
       Then the following two statements are equivalent:
       \leavevmode
       \begin{enumerate}[i)]
           \item $\mathcal{C}$ has a restriction in $\mathcal{S}$.
           \item There are $k$ members $C_1, C_2, \ldots, C_k$ of $\mathcal{C}$ such that the Clutter obtained by $\mathcal{C}$ after restricting $V - \bigcup_{i=1}^k C_i$ is in $\mathcal{S}$.
       \end{enumerate}
   \end{theorem}

   \begin{proof}
       ($\impliedby$) is immediate.
       ($\implies$) Since Clutters have finite ground set, $\mathcal{C}$ has a restriction-minimal restriction.
       Let this restriction be $\mathcal{C}\backslash I / J$.
       Since $\mathcal{S}$ is $k$-unifying, we find $k$ members $C_1', C_2', \ldots, C_k'$ in this restriction, such that $\bigcup_{i=1}^k C_i' = V - (I \cup J)$.
 Let $C_i = C_i' \cup J$.
       That yields $I=V-\bigcup_{i=1}^k C_i$, therefore such $k$ members are found.
   \end{proof}

   \chapter{Intersecting restrictions}
   \begin{proposition}
       Let $k\geq 2$ be an integer.
       The set of $k$-wise intersecting clutters is $(k+1)$-unifying.
   \end{proposition}

   \begin{proof}
       Let $\mathcal{C}$ be $k$-wise intersecting such that no proper restrictionis $k$-wise intersecting.
       We have to show that $\mathcal{C}$ has $(k+1)$ members whose union is the ground set.
       Choose $(k+1)$ members $C_1, C_2, \ldots, C_{k+1}$ of $\mathcal{C}$ such that $|\bigcap_{i=1}^{k+1} C_i|$ is minimal.
       We prove that the union of these members is the ground set $V$.
       Assume this is not the case.
       Then there is a $v \in V$ such that $v \not\in C_i$ for $i=1,2, \ldots, k+1$.
       Consider the restriction obtained by $\mathcal{C}$ after restricting $v$.
       Let this restriction be $\mathcal{C'}=\mathcal{C} \backslash v / J$.
       Note that by the definition of a restriction, we get $J \subseteq \bigcap_{i=1}^{k+1} C_i$.

       If this restriction is not a proper restriction, $\mathcal{C'}$ is trivial.
       That means $\mathcal{C'}=\{\}$ or $\mathcal{C'}=\{\emptyset\}$.
       The first case is not possible because $\{v\}$ would be a cover of $\mathcal{C}$.
       In the second case, there is a member $C^* \in \mathcal{C}$ with $C^* \subseteq J$.

       If the restriction is a proper restriction, $\mathcal{C'}$ is not $k$-wise intersecting. We therefore find $k$ members $C_1',\ldots, C_k'$ of $\mathcal{C'}$ with empty intersection. We get $k$ members $C_1^*, C_2^*,\ldots, C_k^*$ in $\mathcal{C}$ with $\bigcap_{i=1}^k C_i^* \subseteq J$.
       If the restriction was not proper, we take $k-1$ arbitrary further members to get the same result.

       Note, that the intersection of these $k$ members is not empty since $\mathcal{C}$ is $k$-wise intersecting.
       We find an element $u \in \bigcap_{i=1}^k C_i^*$. Since $\{u\}$ is not a cover, we find $C_{k+1}^*$ with $u \not\in C_{k+1}^*$.
       We conclude
       \begin{align*}
           \bigcap_{i=1}^{k+1} C_i^* \subsetneq J \subseteq \bigcap_{i=1}^{k+1} C_i \;,
       \end{align*}
       a contratiction to the minimality assumption of these $(k+1)$ members.
       Therefore, the union of these $(k+1)$ members is the ground set.


   \end{proof}

   \chapter{Dense restrictions}
\end{document}
